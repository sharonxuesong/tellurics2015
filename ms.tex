% Telluric modeling in Keck/HIRES iodine spectra

\documentclass{emulateapj}
\usepackage{amsmath,amssymb}
\usepackage{xspace}
\usepackage{graphicx}
\bibliographystyle{apj}
\usepackage{epstopdf}
\usepackage{graphicx}
\usepackage{epsfig}
\usepackage{natbib}
\citestyle{aa}
\usepackage{verbatim}
\usepackage{morefloats} % somehow need this to have lots of figures
                        % and tables?
% for attractive links:
\usepackage[colorlinks,urlcolor=blue,citecolor=black,linkcolor=blue]{hyperref}
%\usepackage[nolists]{endfloat} % put floats at end - doesn't work
\interfootnotelinepenalty=10000 % Do not want cross-page footnote!

% chinese character for name
\usepackage{CJK}

\def\beq{\begin{equation}}
\def\eeq{\end{equation}}
\def\bcm{\begin{comment}}
\def\ecm{\end{comment}}
\def\mps{m~s$^{-1}$}
\def\msini{M\sin{i}}
\def\mjup{M_{\rm Jup}}
\def\msol{M_{\odot}}

\slugcomment{}
\shorttitle{}
\shortauthors{}

\begin{document}

%%%%%%%%%%%%%%%%%%%%%%%%%%%%%%%%%%%%%%%%%%%%%%%%%%%%%%%%%%%%%%%%%%%%%%%%%%%%

\begin{CJK*}{UTF8}{gbsn}

\title
{
The Effects of Telluric Lines in Radial Velocity Searches for Planets with Iodine
Cell as Calibrators\altaffilmark{1}
}

\altaffiltext{1}
{
Based on observations observations obtained at the Keck Observatory, which is operated
by the University of California.  The Keck Observatory was made
possible by the generous financial support of the W. M. Keck
Foundation.
}

\author{Sharon Xuesong Wang (王雪凇)\altaffilmark{2,3},
  Jason T. Wright\altaffilmark{2,3}} % primary work force
\author{Chad Bender\altaffilmark{2,3}} % lots of help with TERRASPEC
                                % and consultation on telluric lines
\author{Andrew W. Howard\altaffilmark{4}} % Keck
\author{Geoffrey W. Marcy\altaffilmark{5}} % Keck
\author{Howard Isaacson\altaffilmark{5}} % Keck
\author{Suvrath Mahadevan\altaffilmark{2,3}} % comments and suggestions


\altaffiltext{2}{Department of Astronomy and Astrophysics, 525 Davey
  Laboratory, The Pennsylvania State University, University Park, PA
  16802, USA; Send correspondence to xxw131@psu.edu and
  jtwright@astro.psu.edu}

\altaffiltext{3}{Center for Exoplanets and Habitable Worlds, 525 Davey
  Laboratory, The Pennsylvania State University, University Park, PA
  16802, USA}

\altaffiltext{4}{Hawaii, USA}

\altaffiltext{5}{Department of Astronomy, University of California,
  Berkeley, CA 94720, USA}


%%%%%%%%%%%%%%%%%%%%%%%%%%%%%%%%%%%%%%%%%%%%%%%%%%%%%%%%%%%%%%%%%%%%%%%%%%%%
%%%%%%%%%%%%%%%%%%%%%%%%%%%%%%%%%%%%%%%%%%%%%%%%%%%%%%%%%%%%%%%%%%%%%%%%%%%%
\begin{abstract}

Tellurics are bad and you really don't want them. Here's how to get
rid of them.

\end{abstract}

\keywords{instrumentation}


%%%%%%%%%%%%%%%%%%%%%%%%%%%%%%%%%%%%%%%%%%%%%%%%%%%%%%%%%%%%%%%%%%%%%%%%%%%%
%%%%%%%%%%%%%%%%%%%%%%%%%%%%%%%%%%%%%%%%%%%%%%%%%%%%%%%%%%%%%%%%%%%%%%%%%%%%
\section{Introduction}\label{sec:intro}

We are going to cite \cite{artigau2014, cunha2014} and \cite{hitran2013}.


%%%%%%%%%%%%%%%%%%%%%%%%%%%%%%%%%%%%%%%%%%%%%%%%%%%%%%%%%%%%%%%%%%%%%%%%%%%%
%%%%%%%%%%%%%%%%%%%%%%%%%%%%%%%%%%%%%%%%%%%%%%%%%%%%%%%%%%%%%%%%%%%%%%%%%%%%
\section{Simulating Keck/HIRES Spectra}\label{sec:simulating}

Introduction to Keck/HIRES CPS data, data reduction, Doppler code. Our
adoption of Doppler code.

Introduction to the standard star(s) in this paper?

ZZZ Table: standard star stellar properties? esp. coordinates to
highlight BC ranges?


%%%%%%%%%%%%%%%%%%%%%%%%%%%%%%%%%%%%%%%%%%%%%%%%%%%%%%%%%%%%%%%%%%%%%%%%%%%%
%%%%%%%%%%%%%%%%%%%%%%%%%%%%%%%%%%%%%%%%%%%%%%%%%%%%%%%%%%%%%%%%%%%%%%%%%%%%
\section{Injecting Telluric Lines}\label{sec:injecting}


%%%%%%%%%%%%%%%%%%%%%%%%%%%%%%%%%%%%%%%%%%%%%%%%%%%%%%%%%%%%%%%%%%%%%%%%%%%%
\subsection{Simulated Data}

This is where TERRASPEC got introduced.

ZZZ Plots:

- telluric line atlas in the optical? just to demonstrate where they are.

- telluric free simulations to demonstrate RV precision, vs. year,
vs. BC

- telluric injected simulations to demonstrate effects, vs. year, vs. BC

- RV vs. BC for chunks with and without telluric lines before and
after injection


%%%%%%%%%%%%%%%%%%%%%%%%%%%%%%%%%%%%%%%%%%%%%%%%%%%%%%%%%%%%%%%%%%%%%%%%%%%%
\subsection{On-Sky Data}

Should this be the first?

ZZZ Plots:

- RV vs. year/time, RV vs. BC

- RV from chunks with and without telluric lines, vs. BC


%%%%%%%%%%%%%%%%%%%%%%%%%%%%%%%%%%%%%%%%%%%%%%%%%%%%%%%%%%%%%%%%%%%%%%%%%%%%
\subsection{Where and How Telluric Lines Enter the Data}

Should this subsection be moved to the first?
It enters the epoch data as well as the empirically derived stellar template.

ZZZ Plots:

- fitting plots showing data (with telluric highlighted) and DSST
(tellurics highlighted) and the mismatch


%%%%%%%%%%%%%%%%%%%%%%%%%%%%%%%%%%%%%%%%%%%%%%%%%%%%%%%%%%%%%%%%%%%%%%%%%%%%
%%%%%%%%%%%%%%%%%%%%%%%%%%%%%%%%%%%%%%%%%%%%%%%%%%%%%%%%%%%%%%%%%%%%%%%%%%%%
\section{Correcting for the Effects of Telluric Lines}


%%%%%%%%%%%%%%%%%%%%%%%%%%%%%%%%%%%%%%%%%%%%%%%%%%%%%%%%%%%%%%%%%%%%%%%%%%%%
\subsection{Masking out the Telluric Lines}

ZZZ Plot:

- Illustration of choice of mask

- RV vs. BC before and after for all, telluric, and non-telluric chunks

- simulation results showing no aliases introduced by masking


%%%%%%%%%%%%%%%%%%%%%%%%%%%%%%%%%%%%%%%%%%%%%%%%%%%%%%%%%%%%%%%%%%%%%%%%%%%%
\subsection{Removing the Telluric Contamination from Stellar Templates}

ZZZ Plot:

- selected chunk before and after cleaning

- ?RV vs. BC using cleaned template, no masking, single mask?, double mask?


%%%%%%%%%%%%%%%%%%%%%%%%%%%%%%%%%%%%%%%%%%%%%%%%%%%%%%%%%%%%%%%%%%%%%%%%%%%%
\subsection{Modeling Telluric Lines in the Star$+$Iodine Spectra}

ZZZ Plot:

- RV vs. BC using cleaned template plus full telluric modeling

- RV vs. BC using cleaned template plus full modeling plus some masking?


%%%%%%%%%%%%%%%%%%%%%%%%%%%%%%%%%%%%%%%%%%%%%%%%%%%%%%%%%%%%%%%%%%%%%%%%%%%%
\subsection{Results for A Different RV Standard Star}

I think this is a must! At least one more star, hopefully two.


%%%%%%%%%%%%%%%%%%%%%%%%%%%%%%%%%%%%%%%%%%%%%%%%%%%%%%%%%%%%%%%%%%%%%%%%%%%%
%%%%%%%%%%%%%%%%%%%%%%%%%%%%%%%%%%%%%%%%%%%%%%%%%%%%%%%%%%%%%%%%%%%%%%%%%%%%
\section{Summary and Future Work}\label{sec:summary}

Gaussian processes!


%%%%%%%%%%%%%%%%%%%%%%%%%%%%%%%%%%%%%%%%%%%%%%%%%%%%%%%%%%%%%%%%%%%%%%%%%%%%
% Then Acknowledgement
% Thank John Johnson for providing CPS Doppler code.
\acknowledgements
% John
The authors thank John A. Johnson for providing a copy of his Doppler
code and his help with our incorporation of the code into the HET
pipeline.  The authors also thank Debra Fischer for her assistance in
this regard.

% CEHW
This work was partially supported by funding from the Center for
Exoplanets and Habitable Worlds, which is supported by the
Pennsylvania State University, the Eberly College of Science, and the
Pennsylvania Space Grant Consortium.

% Financial support acknowledgement
The authors appreciate the significant Keck observing time and
associated funding support from NASA for the study of long period
planets and mulitplanet systems.
%
J.T.W.\ and S.X.W.\ acknowledge support from NASA Origins of Solar
Systems grant NNX10AI52G.

% Keck
The work herein is based on observations obtained at the W. M. Keck
Observatory, which is operated jointly by the University of California
and the California Institute of Technology.  The Keck Observatory was
made possible by the generous financial support of the W.M. Keck
Foundation.  We wish to recognize and acknowledge the very significant
cultural role and reverence that the summit of Mauna Kea has always
had within the indigenous Hawaiian community.  We are most fortunate
to have the opportunity to conduct observations from this mountain.

% HET
The Hobby-Eberly Telescope is a joint project of the University of
Texas at Austin, the Pennsylvania State University, Stanford
University, Ludwig Maximillians Universit\"at M\"unchen, and Georg
August Universit\"at G\"ottingen. The HET is named in honor of its
principal benefactors, William P. Hobby and Robert E. Eberly.

% ADS
This work has made use NASA’s Astrophysics Data System Bibliographic Services.


\end{CJK*}

%%%%%%%%%%%%%%%%%%%%%%%%%%%%%%%%%%%%%%%%%%%%%%%%%%%%%%%%%%%%%%%%%%%%%%%%%%%%
% The Bibliography
%\bibliographystyle{apj} % (uses file "xxx.bst")
%\bibliography{mn-jour,references}%,bootstrap-ref}
\bibliography{references}
%%%%%%%%%%%%%%%%%%%%%%%%%%%%%%%%%%%%%%%%%%%%%%%%%%%%%%%%%%%%%%%%%%%%%%%%%%%%


\end{document}
