% Telluric modeling in Keck/HIRES iodine spectra

\documentclass{emulateapj}
\usepackage{amsmath,amssymb}
\usepackage{xspace}
\usepackage{graphicx}
\bibliographystyle{apj}
\usepackage{epstopdf}
\usepackage{graphicx}
\usepackage{epsfig}
\usepackage{natbib}
\citestyle{aa}
\usepackage{verbatim}
\usepackage{morefloats} % somehow need this to have lots of figures
                        % and tables?
% for attractive links:
\usepackage[colorlinks,urlcolor=blue,citecolor=black,linkcolor=blue]{hyperref}
%\usepackage[nolists]{endfloat} % put floats at end - doesn't work
\interfootnotelinepenalty=10000 % Do not want cross-page footnote!

% chinese character for name
\usepackage{CJK}

\def\beq{\begin{equation}}
\def\eeq{\end{equation}}
\def\bcm{\begin{comment}}
\def\ecm{\end{comment}}
\def\mps{m~s$^{-1}$}
\def\msini{M\sin{i}}
\def\mjup{M_{\rm Jup}}
\def\msol{M_{\odot}}

\slugcomment{}
\shorttitle{}
\shortauthors{}

\begin{document}

%%%%%%%%%%%%%%%%%%%%%%%%%%%%%%%%%%%%%%%%%%%%%%%%%%%%%%%%%%%%%%%%%%%%%%%%%%%%

\begin{CJK*}{UTF8}{gbsn}

\title
{
The Effects of Telluric Lines in Radial Velocity Searches for Planets with Iodine
Cell as Calibrators\altaffilmark{1}
}

\altaffiltext{1}
{
Based on observations observations obtained at the Keck Observatory, which is operated
by the University of California.  The Keck Observatory was made
possible by the generous financial support of the W. M. Keck
Foundation.
}

\author{Sharon Xuesong Wang (王雪凇)\altaffilmark{2,3},
  Jason T. Wright\altaffilmark{2,3}} % primary work force
\author{Chad Bender\altaffilmark{2,3}} % lots of help with TERRASPEC
                                % and consultation on telluric lines
\author{Andrew W. Howard\altaffilmark{4}} % Keck
\author{Geoffrey W. Marcy\altaffilmark{5}} % Keck
\author{Howard Isaacson\altaffilmark{5}} % Keck
\author{Suvrath Mahadevan\altaffilmark{2,3}} % comments and suggestions


\altaffiltext{2}{Department of Astronomy and Astrophysics, 525 Davey
  Laboratory, The Pennsylvania State University, University Park, PA
  16802, USA; Send correspondence to xxw131@psu.edu and
  jtwright@astro.psu.edu}

\altaffiltext{3}{Center for Exoplanets and Habitable Worlds, 525 Davey
  Laboratory, The Pennsylvania State University, University Park, PA
  16802, USA}

\altaffiltext{4}{Hawaii, USA}

\altaffiltext{5}{Department of Astronomy, University of California,
  Berkeley, CA 94720, USA}



%%%%%%%%%%%%%%%%%%%%%%%%%%%%%%%%%%%%%%%%%%%%%%%%%%%%%%%%%%%%%%%%%%%%%%%%%%%%
\begin{abstract}

Tellurics are bad and you really don't want them. Here's how to get
rid of them.

\end{abstract}

\keywords{instrumentation}

%%%%%%%%%%%%%%%%%%%%%%%%%%%%%%%%%%%%%%%%%%%%%%%%%%%%%%%%%%%%%%%%%%%%%%%%%%%%
\section{Introduction}\label{sec:intro}

I'll keep the first paragraph so there are some references -- also
because it's cool. Jupiter analogs orbiting other stars represent the first signposts of
true Solar System analogs, and the eccentricity distribution of these
planets with $a>3$ AU will reveal how rare or frequent true Jupiter
analogs are. To date, only 9 ``Jupiter analogs" have been
well-characterized in the peer reviewed literature\footnote{HD
  13931$b$ \citep{2010ApJ...721.1467H}, HD 72659$b$
  \citep{2011A&A...527A..63M}, 55 Cnc $d$ \citep{2002ApJ...581.1375M},
  HD 134987$c$ \citep{2010MNRAS.403.1703J}, HD 154345$b$ \citep[but
    with possibility of being an activity cycle-induced
    signal]{2008ApJ...683L..63W}, $\mu$ Ara $c$
  \citep{2007A&A...462..769P}, HD 183263$c$
  \citep{2009ApJ...693.1084W}, HD 187123$c$
  \citep{2009ApJ...693.1084W}, and GJ 832$b$
  \citep{2009ApJ...690..743B}.}  \citep[defined here as $P > 8$ years,
  $4 > M\sin{i} > 0.5\ \mjup$, and $e <
  0.3$;][exoplanets.org]{wrighteod}. As the duration of existing
planet searches approach 10--20 years, more and more Jupiter analogs
will emerge from their longest-observed targets
\citep{2012arXiv1205.2765W, 2012arXiv1205.5835B}.



%----------------------------------------------------------------
% SME Parameters Table
\renewcommand{\arraystretch}{1.2} % more row spacing for the table
\begin{deluxetable}{lc}
\tabletypesize{\scriptsize}
\tablecaption{STELLAR PARAMETERS\label{smetable}}
\tablewidth{180pt}
\tablehead{
  \colhead{Parameter} & \colhead{Value~~}
}
\startdata
~~Spectral type\tablenotemark{a} & K0 V ~~ \\
~~Distance (pc)\tablenotemark{a} & 44.0 $\pm$ 2.1 ~~ \\
~~$V$ & 8.661 $\pm$ 0.013 ~~ \\
%$B-V$           &   ~~ \\ % not seen B-V in Jeff's SME file
~~$T_{\mbox{eff}}$ (K) & 5448 $\pm $44 ~~ \\
~~$\log{g}$ & 4.511 $\pm$ 0.024 ~~ \\
~~$[$Fe/H$]$ & 0.336 $\pm$ 0.030 ~~ \\
%~~$v\sin{i}$ (k\mps) & 0.14 $\pm$ 0.5 ~~ \\ % no detection available
~~BC & -0.144 ~~ \\
~~$M_{\mbox{bol}}$ & 5.301 ~~ \\
~~$L_{\star}$ ($L_{\odot}$) & 0.590 $\pm$ 0.058 ~~ \\
~~$R_{\star}$ ($R_{\odot}$) & 0.901 $\pm$ 0.015 ~~ \\
~~$M_{\star}$ ($\msol$) & 1.000 $\pm$ 0.017 ~~ \\
~~$v\sin{i}$ & $<1$ k\mps ~~ \\
~~Age\tablenotemark{b}  & $\sim 7$ Gyr ~~
%S$_{HK}$         &   ~~ \\
%log$R'_{HK}$     &   ~~ \\
%$P_{rot}$ (days) &
\enddata
\tablenotetext{a}{\cite{esa1997,vanleeuwen2008}.}
\tablenotetext{b}{\cite{svalue2010}, see section.}
\end{deluxetable}
%----------------------------------------------------------------




%%%%%%%%%%%%%%%%%%%%%%%%%%%%%%%%%%%%%%%%%%%%%%%%%%%%%%%%%%%%%%%%%%%%%%%%%%%%
% Then Acknowledgement
% Thank John Johnson for providing CPS Doppler code.
\acknowledgements
% John
The authors thank John A. Johnson for providing a copy of his Doppler
code and his help with our incorporation of the code into the HET
pipeline.  The authors also thank Debra Fischer for her assistance in
this regard.

% CEHW
This work was partially supported by funding from the Center for
Exoplanets and Habitable Worlds, which is supported by the
Pennsylvania State University, the Eberly College of Science, and the
Pennsylvania Space Grant Consortium.

% Financial support acknowledgement
The authors appreciate the significant Keck observing time and
associated funding support from NASA for the study of long period
planets and mulitplanet systems.
%
J.T.W.\ and S.X.W.\ acknowledge support from NASA Origins of Solar
Systems grant NNX10AI52G.

% Keck
The work herein is based on observations obtained at the W. M. Keck
Observatory, which is operated jointly by the University of California
and the California Institute of Technology.  The Keck Observatory was
made possible by the generous financial support of the W.M. Keck
Foundation.  We wish to recognize and acknowledge the very significant
cultural role and reverence that the summit of Mauna Kea has always
had within the indigenous Hawaiian community.  We are most fortunate
to have the opportunity to conduct observations from this mountain.

% HET
The Hobby-Eberly Telescope is a joint project of the University of
Texas at Austin, the Pennsylvania State University, Stanford
University, Ludwig Maximillians Universit\"at M\"unchen, and Georg
August Universit\"at G\"ottingen. The HET is named in honor of its
principal benefactors, William P. Hobby and Robert E. Eberly.

% ADS
This work has made use NASA’s Astrophysics Data System Bibliographic Services.


\end{CJK*}

%%%%%%%%%%%%%%%%%%%%%%%%%%%%%%%%%%%%%%%%%%%%%%%%%%%%%%%%%%%%%%%%%%%%%%%%%%%%
% The Bibliography
%\bibliographystyle{apj} % (uses file "xxx.bst")
%\bibliography{mn-jour,references}%,bootstrap-ref}
\bibliography{references}
%%%%%%%%%%%%%%%%%%%%%%%%%%%%%%%%%%%%%%%%%%%%%%%%%%%%%%%%%%%%%%%%%%%%%%%%%%%%


\end{document}
