% Effects of telluric lines, as demonstrated via simulation

\documentclass{emulateapj}
\usepackage{amsmath,amssymb}
\usepackage{xspace}
\usepackage{graphicx}
\bibliographystyle{apj}
\usepackage{epstopdf}
\usepackage{graphicx}
\usepackage{epsfig}
\usepackage{natbib}
\citestyle{aa}
\usepackage{verbatim}
\usepackage{morefloats} % somehow need this to have lots of figures
                        % and tables?
% for attractive links:
\usepackage[colorlinks,urlcolor=blue,citecolor=black,linkcolor=blue]{hyperref}
%\usepackage[nolists]{endfloat} % put floats at end - doesn't work
\interfootnotelinepenalty=10000 % Do not want cross-page footnote!

% chinese character for name
\usepackage{CJK}

\def\beq{\begin{equation}}
\def\eeq{\end{equation}}
\def\bcm{\begin{comment}}
\def\ecm{\end{comment}}
\def\mps{m~s$^{-1}$}
\def\msini{M\sin{i}}
\def\mjup{M_{\rm Jup}}
\def\msol{M_{\odot}}

\slugcomment{}
\shorttitle{}
\shortauthors{}

\begin{document}

%%%%%%%%%%%%%%%%%%%%%%%%%%%%%%%%%%%%%%%%%%%%%%%%%%%%%%%%%%%%%%%%%%%%%%%%%%%%

\begin{CJK*}{UTF8}{gbsn}

\title
{
The Effects of Telluric Lines in Radial Velocity Searches for Planets with Iodine
Cell as Calibrators\altaffilmark{1}
}

\altaffiltext{1}
{
Based on observations observations obtained at the Keck Observatory, which is operated
by the University of California.  The Keck Observatory was made
possible by the generous financial support of the W. M. Keck
Foundation.
}

\author{Sharon Xuesong Wang (王雪凇)\altaffilmark{2,3},
  Jason T. Wright\altaffilmark{2,3}} % primary work force
\author{Chad Bender\altaffilmark{2,3}} % lots of help with TERRASPEC
                                % and consultation on telluric lines
\author{Andrew W. Howard\altaffilmark{4}} % Keck
\author{Geoffrey W. Marcy\altaffilmark{5}} % Keck
\author{Howard Isaacson\altaffilmark{5}} % Keck
\author{Suvrath Mahadevan\altaffilmark{2,3}} % comments and suggestions


\altaffiltext{2}{Department of Astronomy and Astrophysics, 525 Davey
  Laboratory, The Pennsylvania State University, University Park, PA
  16802, USA; Send correspondence to xxw131@psu.edu and
  jtwright@astro.psu.edu}

\altaffiltext{3}{Center for Exoplanets and Habitable Worlds, 525 Davey
  Laboratory, The Pennsylvania State University, University Park, PA
  16802, USA}

\altaffiltext{4}{Hawaii, USA}

\altaffiltext{5}{Department of Astronomy, University of California,
  Berkeley, CA 94720, USA}


%%%%%%%%%%%%%%%%%%%%%%%%%%%%%%%%%%%%%%%%%%%%%%%%%%%%%%%%%%%%%%%%%%%%%%%%%%%%
%%%%%%%%%%%%%%%%%%%%%%%%%%%%%%%%%%%%%%%%%%%%%%%%%%%%%%%%%%%%%%%%%%%%%%%%%%%%
\begin{abstract}

Tellurics are bad and you really don't want them. Here's why and how to get
rid of them.

\end{abstract}

\keywords{instrumentation}


%%%%%%%%%%%%%%%%%%%%%%%%%%%%%%%%%%%%%%%%%%%%%%%%%%%%%%%%%%%%%%%%%%%%%%%%%%%%
%%%%%%%%%%%%%%%%%%%%%%%%%%%%%%%%%%%%%%%%%%%%%%%%%%%%%%%%%%%%%%%%%%%%%%%%%%%%
\section{Introduction}\label{sec:intro}

The radial velocity method has contributed to hundreds of planetary
discoveries and Kepler follow-ups. The current precision is at 1--2
m/s, with limiting factor being stellar activity and RV systematics,
instrumental or software.

One of the contributing term comes from Earth's atmospheric
absorption, which imprints telluric lines onto the observed stellar
spectrum. Deep tellurics are thrown out, while micro-tellurics
(define; \citealt{hitran2013}) remain and they are very hard to deal
with. The contamination from the micro-telluric lines has an adverse
effect on the RV precision, and such effect is characterized for the
HARPS-N spectrograph, which is ThAr calibrated, by \cite{artigau2014}
and \cite{cunha2014}.

In this paper, we characterize and discuss the remedies for
micro-telluric lines in the context of iodine-calibarted precise
RVs. We first quantify the effects of micro-telluric contamination on
RV precision through simulation. Then we discuss possible remedies and
their effectiveness. Finally we discuss implementation on real data
and future work.

\begin{comment}
Jason's suggestions:
- run tau Ceti as target #2
- don't show our classic before+after+difference RV vs. BC, just show
the difference, especially the talk
- show actual data for sig Dra RV vs. BC of Keck, to show that this is
real, it should be standard to plot RV vs. BC to check systematics
(i.e. tellurics)
- Mention Vogt APF paper sig Dra planet.
\end{comment}


%%%%%%%%%%%%%%%%%%%%%%%%%%%%%%%%%%%%%%%%%%%%%%%%%%%%%%%%%%%%%%%%%%%%%%%%%%%%
%%%%%%%%%%%%%%%%%%%%%%%%%%%%%%%%%%%%%%%%%%%%%%%%%%%%%%%%%%%%%%%%%%%%%%%%%%%%
\section{Impacts of Micro-tellurics on RV Precision}\label{sec:method}

We performed end-to-end simulation of Keck data and analysis process
to access the impacts of micro-tellurics on RV precision. We use Keck
to demonstrate this because Keck has the highest precision. We chose
sig Dra and tau Ceti as our stars because they are RV standards which
have been observed hundreds of times with Keck/HIRES, and are also
favorite RV standards at other precise RV facilities. I really want to
add an M dwarf standard here as well!


%%%%%%%%%%%%%%%%%%%%%%%%%%%%%%%%%%%%%
\subsection{Methodology}

We simulated Keck observations on sig Dra and tau Ceti by using
synthetic stellar spectra of their respective spectral types (?) using
SME (ZZZ cite Valenti and Fischer). We simulated one spectrum for each
actual observed spectrum taken at Keck through the CPS programs. The
synthetic stellar spectra is multiplied with the iodine atlas to
create the standard iodine$+$ star RV observations. The multiplied
spectrum is then multiplied with the blaze function and convolved with
the observed spectral PSF, both derived from real observations for
each night. Poisson noise is added.

We then forward model the simulated spectra to extract RVs using the
CPS Keck code (ZZZ cite Johnson and Howard). We used the synthetic
stellar spectrum as the input stellar template. In reality, stellar
templates are derived from observed stellar spectra via deconvolution,
which would introduce additional errors. Using the same synthetic
stellar spectrum would eliminate such errors and isolate the problem
to telluric lines only.

We ran two sets of simulations: control and contaminated. In the
control, we only had stellar spectrum and iodine spectrum. In the
contaminated, we added in simulated telluric lines in the simulated
observed spectrum. The telluric lines were generated using TERRASPEC
(ZZZ cite Bender). We adopted the typical Mauna Kea atmospheric
condition (temperature and pressure profiles) and typical oxygen
column density (which in realiaty flucturate very little anyway). For
simplicity, we assumed the same water column density for every
observation, which is pwv$=1$mm, a little bit humid than a typical
Mauna Kea night (true? I think this is actually pretty typical). The
pair of simulated control and contaminated spectra have the same added
Poisson noise, and therefore any RV differences derived from these two
sets of simulation would reveal the net effect of telluric
contamination.


%%%%%%%%%%%%%%%%%%%%%%%%%%%%%%%%%%%%%
\subsection{Results}

Plots: RV difference vs. BC for sig Dra and tau Ceti. And M stars.

Micro-tellurics in the iodine region introduces RMS$=0.6$ m/s scatter
for GK stars (RV systematic error added in quadrature). Leaving
untreated, this would define the precision floor.

Additionally, it manifests as spurious signal at periods of a sidereal
year and harmonics, with an amplitude of 20 cm/s. This would affect
our ability to detect super-Earth in the habitable zone of GK stars
(Earth's signal is 8 cm/s). We have seen such spurious signal in Keck
data on many stars, and telluric contamination is one of the
contributing factors (see discussion for other factors).

For M stars... (probably worse)


%%%%%%%%%%%%%%%%%%%%%%%%%%%%%%%%%%%%%%%%%%%%%%%%%%%%%%%%%%%%%%%%%%%%%%%%%%%%
%%%%%%%%%%%%%%%%%%%%%%%%%%%%%%%%%%%%%%%%%%%%%%%%%%%%%%%%%%%%%%%%%%%%%%%%%%%%
\section{Remedies and Effectiveness}

Plots: effects of full forward modeling (RV difference between modeled
and contaminated, modeled and control).

We demonstrate the effectiveness of several remedies. First, double
masking: probably a terrible idea. Describe the effects
quantitatively. Second, modeling plus cleaning tempaltes: probably
effective. Modeling plus some masking?



%%%%%%%%%%%%%%%%%%%%%%%%%%%%%%%%%%%%%%%%%%%%%%%%%%%%%%%%%%%%%%%%%%%%%%%%%%%%
%%%%%%%%%%%%%%%%%%%%%%%%%%%%%%%%%%%%%%%%%%%%%%%%%%%%%%%%%%%%%%%%%%%%%%%%%%%%
\section{Discussion}



ZZZ Plots:

- demonstration of effectiveness and ineffectiveness of double masking.

- demonstration of effectiveness of a clean template and forward
modeling tellurics.

- demonstration of precision required for modeling tellurics, and
maximum tolerance for template `cleaness'. 


%%%%%%%%%%%%%%%%%%%%%%%%%%%%%%%%%%%%%%%%%%%%%%%%%%%%%%%%%%%%%%%%%%%%%%%%%%%%
%%%%%%%%%%%%%%%%%%%%%%%%%%%%%%%%%%%%%%%%%%%%%%%%%%%%%%%%%%%%%%%%%%%%%%%%%%%%
\section{Summary and Future Work}\label{sec:summary}

Work on actual observations!


%%%%%%%%%%%%%%%%%%%%%%%%%%%%%%%%%%%%%%%%%%%%%%%%%%%%%%%%%%%%%%%%%%%%%%%%%%%%
% Then Acknowledgement
% Thank John Johnson for providing CPS Doppler code.
\acknowledgements
% John
The authors thank John A. Johnson for providing a copy of his Doppler
code and his help with our incorporation of the code into the HET
pipeline.  The authors also thank Debra Fischer for her assistance in
this regard.

% CEHW
This work was partially supported by funding from the Center for
Exoplanets and Habitable Worlds, which is supported by the
Pennsylvania State University, the Eberly College of Science, and the
Pennsylvania Space Grant Consortium.

% Financial support acknowledgement
The authors appreciate the significant Keck observing time and
associated funding support from NASA for the study of long period
planets and mulitplanet systems.
%
J.T.W.\ and S.X.W.\ acknowledge support from NASA Origins of Solar
Systems grant NNX10AI52G.

% Keck
The work herein is based on observations obtained at the W. M. Keck
Observatory, which is operated jointly by the University of California
and the California Institute of Technology.  The Keck Observatory was
made possible by the generous financial support of the W.M. Keck
Foundation.  We wish to recognize and acknowledge the very significant
cultural role and reverence that the summit of Mauna Kea has always
had within the indigenous Hawaiian community.  We are most fortunate
to have the opportunity to conduct observations from this mountain.

% HET
The Hobby-Eberly Telescope is a joint project of the University of
Texas at Austin, the Pennsylvania State University, Stanford
University, Ludwig Maximillians Universit\"at M\"unchen, and Georg
August Universit\"at G\"ottingen. The HET is named in honor of its
principal benefactors, William P. Hobby and Robert E. Eberly.

% ADS
This work has made use NASA’s Astrophysics Data System Bibliographic Services.


\end{CJK*}

%%%%%%%%%%%%%%%%%%%%%%%%%%%%%%%%%%%%%%%%%%%%%%%%%%%%%%%%%%%%%%%%%%%%%%%%%%%%
% The Bibliography
%\bibliographystyle{apj} % (uses file "xxx.bst")
%\bibliography{mn-jour,references}%,bootstrap-ref}
\bibliography{references}
%%%%%%%%%%%%%%%%%%%%%%%%%%%%%%%%%%%%%%%%%%%%%%%%%%%%%%%%%%%%%%%%%%%%%%%%%%%%


\end{document}
